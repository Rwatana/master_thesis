\chapter{導入}

\section{概要}
近年,スマートフォンとSNSの普及に伴い,デジタルマーケティングは急速に拡大している。中でも,SNS上で影響力を持つ個人(インフルエンサー)を介して情報を届けるインフルエンサーマーケティングは,企業の認知獲得やブランディング,購買行動の喚起において重要な施策として位置付けられている \cite{Lou2019,Campbell2020,Hudders2021}。

マーケティング実務における重要課題の一つは,限られた予算の中で「誰に依頼すべきか」を意思決定することである。しかし,著名なトップインフルエンサーは依頼費用が高額であり,競合との獲得競争も激しい \cite{Haenlein2020}。そこで近年は,現時点ではフォロワー規模が大きくなくとも将来的に伸長が見込まれる人材を早期に発掘し,長期的関係を構築する選定(以下,青田買い)が注目されている \cite{Hudders2021}。

将来伸長するインフルエンサーの発掘には,投稿頻度,投稿内容(テキストや画像),ハッシュタグ選択,他者との関係性(メンションや共起)など,多様な要因が複合的に関与する \cite{Casalo2020}。さらに,これらは時系列に沿って変化し,SNS上の関係構造も動的に推移する。そのため,単純な静的指標(フォロワー数等)だけでは将来の影響力を捉えにくい。また,不正アカウントや購入フォロワー等により指標が歪む可能性も指摘されている \cite{Cresci2015}。

このような背景のもと,異種混合グラフと時系列学習を組み合わせ,将来伸長が見込まれる有望なインフルエンサーをランキングする手法が提案されている。例えば InfluencerRank は,各期間の異種グラフをグラフ畳み込み(GCN)で表現し,その推移を再帰モデルと注意機構で統合することで,将来の影響力を推定する \cite{Kim2023InfluencerRank}。一方で,深層学習に基づく予測はブラックボックスになりやすく,実務上は「誰が伸びるか」だけでなく,「なぜ伸びると判断したか」という根拠提示が,投資判断・説明責任・施策設計の観点から重要である \cite{Gunning2019,Arrieta2020XAI,Liao2020QuestioningAI,Rai2020GlassBox}。

本研究では,InfluencerRank 型のマルチモーダル時系列グラフ予測モデルに対して,最適化ベースのマスク学習により説明を与える枠組みを検討する。具体的には,予測をできるだけ維持しつつ入力(ノード特徴量およびエッジ)を疎に残すマスクを学習し,予測に寄与する要素をコンパクトに抽出する。これは GNNExplainer に代表される,予測と説明の両立(忠実性と簡潔性のトレードオフ)に基づく考え方である \cite{Ying2019GNNExplainer}。本枠組みにより,画像・テキスト・数値特徴量とグラフ構造の双方について,どの要素が将来スコアに寄与したかを月次で比較可能な形で提示することを目指す。

また,実務では「いいね数の非表示」等のように可観測な評価指標が変化し得る。これに対し本研究の枠組みは,過去のエンゲージメント値そのものを入力として用いず,投稿コンテンツ(画像・テキスト)や行動・関係性といった情報から将来指標を推定することを重視する。この点は,観測可能な指標や表示仕様が変化しても,コンテンツとネットワーク形成に基づく評価・予測へ接続しやすいという実務上の利点に繋がると考えられる。

本研究のアプローチは以下の2点の統合からなる。
\begin{enumerate}
  \item 異種混合グラフと時系列モデル(本研究では LSTM)による将来エンゲージメント推定と,インフルエンサーノードのランキング(InfluencerRank 型モデル) \cite{Kim2023InfluencerRank}
  \item 最適化に基づくマスク学習により,ノード特徴量およびエッジの寄与を抽出し,さらに予測スコアの変化に基づく符号付き影響度を定義して月次比較可能な形で提示する説明手法 \cite{Ying2019GNNExplainer}
\end{enumerate}

\section{本研究の目的}
本研究の目的は,インフルエンサーマーケティングにおける青田買い意思決定を支援するために,将来予測の精度と説明可能性を両立する枠組みを構築することである。
具体的には,(i) マルチモーダルかつ時系列・グラフ構造を含む入力から将来の影響力スコアを推定し,(ii) その推定に対して,どの特徴量・どの関係性がどの時点で効いたのかを,予測に整合する形で提示する。
本研究では影響力(エンゲージメント)を,いいね数とコメント数の合計をフォロワー数で正規化した指標として扱い,ランキング対象はインフルエンサーノードに限定する。
これにより,候補者選定の納得感を高めるとともに,施策(投稿頻度の改善,ハッシュタグ方針,コラボ関係の設計等)に繋げられる分析基盤の提供を目指す \cite{Rai2020GlassBox,Liao2020QuestioningAI}。

\section{本論文の構成}
本論文は全9章から構成される。本文中では,章・節・小節の番号をそれぞれ
「章番号」,「章番号.節番号」,「章番号.節番号.小節番号」(例:2.1,2.3.1)の形式で表記する。

第2章では,インフルエンサーマーケティングにおけるインフルエンサー選定の背景と課題を整理した上で,
「将来伸長するインフルエンサー」を予測する既存研究を概観する。特に,InfluencerRank 型モデルについて,
異種混合グラフの表現,GCN による各期間グラフの表現学習,時系列モデルと注意機構による統合,
ランキング学習および評価指標の位置づけを述べる \cite{Kim2023InfluencerRank}。

第3章では,本研究の問題設定を与える。入力として用いるデータ(各時点の異種混合グラフ,ノード特徴量,エッジ集合,
対象とするインフルエンサーノード,観測窓長など)と,出力として扱う量(将来時点の影響力スコアおよびランキング)を定式化する。
また,説明の対象(どの時点のどの要素を説明するか)と,説明に求める要件(忠実性・簡潔性・比較可能性)を整理する。

第4章では,先行研究および本研究のベースとなる説明手法を整理し,時系列グラフ予測モデルに適用可能な
最適化ベースのマスク学習手法を述べる。具体的には,時点ごとのマスク付与,直列モデル
(GCN $\rightarrow$ 時系列モデル/Attention $\rightarrow$ MLP)に対する説明の設計,
月次比較を行うための集約や正規化の方針を示す。さらに,特徴量とエッジの両方について,
予測スコアの変化に基づく符号付き影響度を定義し,その計算手順と解釈を述べる。

第5章では,実データを用いた追試実験を行い,予測性能(ランキング指標)と説明結果を評価する。
どの特徴量群およびどの関係性が予測に寄与したか,それらが月次でどのように変化したか,
重要度と符号付き影響度が一致・乖離する状況は何か,といった観点から結果を整理する。
加えて,反実仮想的な検証や安定性の観点から説明の妥当性を議論する。

第6章では,本研究の限界と今後の展望を述べる。マスク最適化に内在する不安定性や多解性,
ベースライン設定や正則化の影響,時系列比較におけるスケーリングの課題,因果性との混同リスクなどを整理し,
改善の方向性を議論する。

第7章では考察として,本研究で得られた知見を実務上の意思決定へ接続する観点からまとめる。

第8章では結論として,本研究の貢献を総括する。

第9章では謝辞を述べる。
