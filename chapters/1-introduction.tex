\chapter{導入}

\section{概要}
近年,スマートフォンとSNSの普及に伴い,デジタルマーケティングは急速に拡大している。中でも,SNS上で影響力を持つ個人(インフルエンサー)を介して情報を届けるインフルエンサーマーケティングは,企業の認知獲得やブランディング,購買行動の喚起において重要な施策として位置付けられている \cite{Lou2019,Campbell2020,Hudders2021}。

マーケティング実務における重要課題の一つは,限られた予算の中で「誰に依頼すべきか」を意思決定することである。しかし,著名なトップインフルエンサーは依頼費用が高額であり,競合との獲得競争も激しい \cite{Haenlein2020}。そこで近年は,現時点ではフォロワー規模が大きくなくとも将来的に伸長が見込まれる人材を早期に発掘し,長期的関係を構築する選定(以下,青田買い)が注目されている \cite{Hudders2021}。

将来伸長するインフルエンサーの発掘には,投稿頻度,投稿内容(テキストや画像),ハッシュタグ選択,他者との関係性(メンションや共起)など,多様な要因が複合的に関与する \cite{Casalo2020}。さらに,これらは時系列に沿って変化し,SNS上の関係構造も動的に推移する。そのため,単純な静的指標(フォロワー数等)だけでは将来の影響力を捉えにくい。また,不正アカウントや購入フォロワー等により指標が歪む可能性も指摘されている \cite{Cresci2015}。

このような背景のもと,異種混合グラフと時系列学習を組み合わせ,将来伸長が見込まれる有望なインフルエンサーをランキングする手法が提案されている。例えば InfluencerRank は,各期間の異種グラフをグラフ畳み込み(GCN)で表現し,その推移を再帰モデルと注意機構で統合することで,将来の影響力を推定する \cite{Kim2023InfluencerRank}。一方で,深層学習に基づく予測はブラックボックスになりやすく,実務上は「誰が伸びるか」だけでなく,「なぜ伸びると判断したか」という根拠提示が,投資判断・説明責任・施策設計の観点から重要である \cite{Gunning2019,Arrieta2020XAI,Liao2020QuestioningAI,Rai2020GlassBox}。

本研究では,InfluencerRank 型のマルチモーダル時系列グラフ予測モデルに対して,最適化ベースのマスク学習により説明を与える枠組みを検討する。具体的には,予測をできるだけ維持しつつ入力(ノード特徴量およびエッジ)を疎に残すマスクを学習し,予測に寄与する要素をコンパクトに抽出する。これは GNNExplainer に代表される,予測と説明の両立(忠実性と簡潔性のトレードオフ)に基づく考え方である \cite{Ying2019GNNExplainer}。本枠組みにより,画像・テキスト・数値特徴量とグラフ構造の双方について,どの要素が将来スコアに寄与したかを月次で比較可能な形で提示することを目指す。

マスクによる重要度が実際の予測変化との相関を調査し、忠実性を検証する。また,重要月の特定や月次比較が実際に機能しているかも確認する。これにより,インフルエンサーマーケティングにおける青田買い意思決定を支援するための,将来予測の精度と説明可能性を両立する枠組みを構築することを目的とする。また、特徴量がインフルエンサー全体に共通して効くのか、特定ユーザーにのみ効くのかを区別できるかも検証する。

また,現時点では未実装だが、実務では「いいね数の非表示」等のように可観測な評価指標が変化し得る。これに対し本研究の枠組みは,過去のエンゲージメント値そのものを入力として用いず,投稿コンテンツ(画像・テキスト)や行動・関係性といった情報から将来指標を推定することを重視する。この点は,観測可能な指標や表示仕様が変化しても,コンテンツとネットワーク形成に基づく評価・予測へ接続しやすいという実務上の利点に繋がると考えられる。

本研究のアプローチは以下の2点の統合からなる。
\begin{enumerate}
  \item 異種混合グラフと時系列モデル(本研究では LSTM)による将来エンゲージメント推定と,インフルエンサーノードのランキング(InfluencerRank 型モデル) \cite{Kim2023InfluencerRank}
  \item 最適化に基づくマスク学習により,ノード特徴量およびエッジの寄与を抽出し,さらに予測スコアの変化に基づく符号付き影響度を定義して月次比較可能な形で提示する説明手法 \cite{Ying2019GNNExplainer}
\end{enumerate}

\section{本研究の目的}
本研究の目的は,インフルエンサーマーケティングにおける青田買い意思決定を支援するために,将来予測の精度と説明可能性を両立する枠組みを構築することである。

具体的には,(i) マルチモーダルかつ時系列・グラフ構造を含む入力から将来の影響力スコアを推定し,(ii) その推定に対して,どの特徴量・どの関係性がどの時点で効いたのかを,予測に整合する形で提示する。
本研究では影響力(エンゲージメント)を,いいね数とコメント数の合計をフォロワー数で正規化した指標として扱い,ランキング対象はインフルエンサーノードに限定する。
これにより,候補者選定の納得感を高めるとともに,施策(投稿頻度の改善,ハッシュタグ方針,コラボ関係の設計等)に繋げられる分析基盤の提供を目指す \cite{Rai2020GlassBox,Liao2020QuestioningAI}。

% ===== 1.2 末尾に追記:研究質問と貢献(貼り付け可) =====
本研究では,説明を「それっぽい可視化」ではなく,少なくとも以下の観点で検証可能な対象として扱う:

RQ1(予測再現): InfluencerRank 型モデルは,本研究の実装・設定においても妥当なランキング性能を示すか.

RQ2(忠実性): マスク最適化で得た重要度は,実際に入力要素を操作したときの予測変化(score impact)と整合するか.

RQ3(比較可能性): 説明は月次で比較可能な形(どの月が効いたか/どの要素が効いたか)として安定に提示できるか.

RQ4(一般性): 重要とされる要素は「多くのインフルエンサーに共通して効く」のか「特定ユーザーにのみ効く」のかを区別できるか.

本研究の主な貢献は以下の 3 点である:
(1) 異種混合時系列グラフに基づく影響力予測モデル(InfluencerRank)の追試実装と,学習・評価設定の整理.
(2) 予測モデルに対して,特徴量・エッジの双方を対象とした End-to-End マスク最適化(MaskOpt)を適用し,重要度(importance)と符号付き影響度(score impact)を分離して提示する説明枠組みの提案.
(3) 説明の妥当性を,スコア差分・ランダム化テスト・集約ヒートマップ等により多面的に検証し,「何がどの程度言えるか/言えないか」を明確化した点.


\section{本論文の構成}
% ===== 1.3 本論文の構成(全文差し替え) =====
本論文は全 9 章から構成される.
第 2 章では,本研究の理解に必要な前提として,GNN,時系列モデルと Temporal Attention,ならびに GNN に対する説明可能性(マスク最適化系)の基本概念と評価指標を整理する.
第 3 章では,本研究が扱うデータセットと予測タスクを定義し,入力(各時点の異種混合グラフとノード特徴量)および出力(将来時点の影響力スコアとランキング)を定式化する.加えて,データセットの妥当性と限界(フォロー関係欠如等)も明示する.
第 4 章では,先行研究として InfluencerRank のモデル構造と学習方法を整理し,比較対象となる GNNExplainer 系手法の枠組みを述べる.
第 5 章では,本研究の提案手法として,時系列グラフ予測モデルに対する End-to-End マスク最適化(MaskOpt)を定式化し,(i) どの時刻を説明対象とするか,(ii) 重要度(importance)と符号付き影響度(score impact)をどのように算出・解釈するか,(iii) GNNExplainer との差分を明確化する.
第 6 章では,RQ に沿って実験を行う.具体的には,予測性能の確認(RQ1),重要度と score impact の整合性検証(RQ2),月次比較可能性や安定性の検証(RQ3),さらに集約ヒートマップ等による一般性の検討(RQ4)を行う.
第 7 章では,実験結果を踏まえ,手法の限界(多解性,ベースライン依存性,サンプル数制約,交互作用の扱い等)と今後の改善方向を議論する.
第 8 章では結論として,本研究で得られた知見を総括する.第 9 章では謝辞を述べる.
