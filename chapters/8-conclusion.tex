\chapter{結論}
本研究では,将来有望なインフルエンサーを早期に発掘する青田買いタスクを対象に,
マルチモーダルな情報(ノード特徴量とグラフ構造)を入力とする時系列グラフ予測モデルにより,
将来月のエンゲージメント(影響力)を予測する枠組みを検討した.
さらに,学習済み予測器を固定したまま最適化ベースのマスク学習(GNNExplainer 系)を適用し,
予測根拠を特徴量およびエッジの観点から抽出・検証する手順を提示した.

実験の結果,マスク最適化により得られる重要度(importance)を可視化することで,
月ごとに上位へ集中する要素が現れ,候補要素を top-$k$ や閾値で限定できる場合があることを確認した.
このことは,説明を重要候補の絞り込みとして用いる観点で有用であり,
高次元な特徴量や多数のエッジを含む設定において,人手分析の探索範囲を削減できる可能性を示す.
一方で,月によってはマスク値が0付近に集中しすぎる傾向や,
最適化が安定しない可能性も観測され,マスク値を精密な定量量として解釈するには注意が必要であることが示唆された.
そこで本研究では,重要度のみで結論を出すのではなく,
要素の置換・除去によるスコア差分(score impact)を併用し,
重要度と実際の影響の整合性を確認することの必要性を整理した.

以上より,本研究の貢献は,インフルエンサー予測モデルに対して,
学習不要な検証(入力介入)に基づく説明の評価手順を与え,
予測結果に対する説明を候補絞り込みと整合性チェックの形で提示可能であることを示した点にある.
これにより,意思決定者は単なるランキング出力だけでなく,
どの要素が予測に関連しているかを参照しつつ分析を進めるための足場を得られる.

今後の課題としては,第一に,説明の安定性と一般性の検証が挙げられる.
複数ユーザー・複数月への適用を拡大し,初期値や正則化条件に対する頑健性,
top-$k$ の一致率や削除曲線などの評価指標に基づき,再現性を確認する必要がある.
第二に,単一要素の差分では扱いにくい相互作用(交互作用)を考慮した貢献度設計が課題である.
第三に,グラフ構造由来の要因をより信頼できる形で扱うため,
エッジ(関係性)に対する影響度の検証手順を拡張し,
ノード特徴量とネットワーク構造の双方から成長メカニズムを説明できる枠組みへ発展させる必要がある.
