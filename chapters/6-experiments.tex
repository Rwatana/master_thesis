\chapter{実験}

\section{実験の目的}
本研究では,学習済みの時系列グラフ予測モデルに対して,最適化ベースのマスク学習により得られる説明が妥当な振る舞いをするかを検証する.
ここで妥当性とは,重要とされた要素(特徴量・エッジ)を操作すると予測が想定通りに変化すること,および説明が少数要素にまとまり月次で比較可能な形で提示できることを指す.

本章の目的は以下の2点である.
\begin{enumerate}
  \item 予測性能の定量評価(学習済みモデルの確認):
  学習済みモデルが,将来月の影響力ランキングをどの程度整合的に出力できているかを,ランキング指標(NDCG,RBP 等)で評価する.
  \item 説明(貢献度)の妥当性検証(学習なしで可能な検証):
  マスク最適化で得られた重要度(importance)と,アブレーションで得る符号付き影響度(score impact)が,予測スコアの変化と整合するかを検証する.
\end{enumerate}

なお,本章では再学習を行わない.すなわち,学習済みモデル $f_\Theta$ を固定し,推論・説明計算・入力操作(アブレーション)により評価を行う.


\section{実験設定}
\subsection{使用データセットと対象ノード}
Instagram データセットを用い,各月 $t$ ごとに異種混合グラフ $G_t=(V,E_t)$ とノード特徴 $X_t$ を構築する.
ランキング対象はインフルエンサーノードに限定する(第3章の問題設定に従う).

影響力(正解指標)は,対象インフルエンサー $u$ の月 $t$ におけるエンゲージメント率として
\begin{align}
y_u^t = \frac{\mathrm{likes}_u^t + \mathrm{comments}_u^t}{\mathrm{followers}_u^t}
\label{eq:eng_rate_eval}
\end{align}
を用いる(実装では月内集計値から算出する).

\subsection{学習済みモデル(固定)}
学習済みモデル $f_\Theta$ は,月次グラフに対するGNNと,時系列統合(LSTM)およびAttentionを直列に接続し,最終的にスコアを出力する.
本章では,既に得られている学習済み重み $\Theta$ を用い,追加の学習は行わない.

\begin{figure}[htbp]
  \centering
  \includegraphics[width=0.85\linewidth]{figures/lstm-attention-mlp.png}
  \caption{実験で用いる学習済み予測器の概観(GNN $\rightarrow$ LSTM $\rightarrow$ Attention $\rightarrow$ MLP)}
  \label{fig:exp_model_overview}
\end{figure}

\subsection{説明(貢献度)算出の定義}
対象インフルエンサー $u$ の予測スコアを
\[
\hat{y}(u)=f_\Theta(\{(G_t,X_t)\}_{t=1}^{T};u)
\]
とする.提案手法により,説明対象月 $\tau$ に対して特徴量マスク $m^X$ とエッジマスク $m^E$ を学習し,次を算出する:
\begin{itemize}
  \item importance(重要度):マスク値そのもの(例:$m^X_j$,$m^E_e$)を重要度として用いる.値が大きいほど,予測を維持するために残したい度合いが高いと解釈する.
  \item score impact(符号付き影響度):ある要素をベースライン置換(特徴)または除去(エッジ)したときのスコア差で定義する.特徴量 $j$ について,
  \begin{align}
    \Delta^X_j(u;\tau) = \hat{y}(u) - f_\Theta(\dots, X^{(j\leftarrow b)}_\tau, \dots;u),
    \label{eq:score_impact_feature}
  \end{align}
  とする.ここで $X^{(j\leftarrow b)}_\tau$ は,月 $\tau$ の対象ノードの特徴 $j$ をベースライン $b_j$ に置換した入力である.
  エッジ $e$ についても同様に,月 $\tau$ のグラフから $e$ を除去した入力を用いて $\Delta^E_e(u;\tau)$ を定義する.
\end{itemize}

重要度は選択強度,影響度は実際にスコアがどれだけ動くか(符号付き)であり,両者がどの程度整合するかを検証する.
また,月を跨いだ比較では重要度(mask値)の絶対値比較は慎重に扱い,主に影響度(スコア差)に基づいて月次比較を行う.


\section{説明の妥当性検証:学習なしで可能な検証}
本節では,学習済みモデルを固定したまま,説明(importance / score impact)が妥当な振る舞いをするかを検証する.
検証は,実験1(マスク値の確認)と実験2(重要度と実影響の整合性チェック)からなる.


\section{実験1:マスク値の確認(1月,2月,3月)}
\subsection{目的}
マスク最適化で得られるマスク値が,説明として利用可能な分布とスケールを持つかを確認する.
具体的には,(i) 上位要素が少数に集中しているか,(ii) 多数の要素が0付近に張り付く(飽和する)傾向がないか,
(iii) 月ごとに分布がどの程度変動するかを観察し,以降の整合性評価の前提を明確にする.

\subsection{手順}
\begin{enumerate}
  \item 対象インフルエンサー $u$ を選択する(例:将来月の予測ランキング上位).
  \item 説明対象月 $\tau$ を 1月,2月,3月に設定し,各 $\tau$ に対して特徴量マスク $m^X$ とエッジマスク $m^E$ を得る.
  \item 各マスクを降順ソートし,top要素を抽出して可視化する(ノード特徴量とエッジを別々に表示する).
\end{enumerate}

\subsection{結果}
図\ref{fig:exp1_jan}〜図\ref{fig:exp1_mar}に,1月,2月,3月それぞれに対する重要度(importance)の可視化結果を示す.
これらの図は,マスク最適化により得られた重要度を上位からプロットし,値の分布がどの程度広がるかを確認する目的で作成したものである.

\begin{figure}[htbp]
  \centering
  \includegraphics[width=0.95\linewidth]{figures/exp1_jan.png}
  \caption{実験1の結果(1月):ノード特徴量とエッジのimportanceの上位プロット}
  \label{fig:exp1_jan}
\end{figure}

\begin{figure}[htbp]
  \centering
  \includegraphics[width=0.95\linewidth]{figures/exp1_feb.png}
  \caption{実験1の結果(2月):ノード特徴量とエッジのimportanceの上位プロット}
  \label{fig:exp1_feb}
\end{figure}

\begin{figure}[htbp]
  \centering
  \includegraphics[width=0.95\linewidth]{figures/exp1_mar.png}
  \caption{実験1の結果(3月):ノード特徴量とエッジのimportanceの上位プロット}
  \label{fig:exp1_mar}
\end{figure}


\section{実験2:重要度と実影響の整合性チェック(1月,2月,3月)}
\subsection{目的}
重要度(importance)が高い要素ほど,その要素を操作したときのスコア変化(score impact)が大きい傾向にあるかを検証する.
これは「説明が予測に忠実である」ことの,学習不要な検証(入力介入による検証)である.

\subsection{手順}
\begin{enumerate}
  \item 実験1と同一の対象インフルエンサー $u$ を用い,説明対象月 $\tau$ を 1月,2月,3月に設定する.
  \item 各 $\tau$ に対して,特徴量とエッジのimportanceを得る.
  \item 特徴量はベースライン置換,エッジは除去を行い,score impact(符号付き影響度)を算出する(式\eqref{eq:score_impact_feature}).
  \item importanceとscore impactの関係を可視化し,整合性を確認する.
\end{enumerate}

\subsection{結果}
図\ref{fig:exp2_jan}〜図\ref{fig:exp2_mar}に,1月,2月,3月それぞれにおけるimportanceとscore impactの関係を示す.
本図は,重要度が高いとされた要素が,実際にスコアに影響する要素として振る舞っているかを確認する目的で作成したものである.

\begin{figure}[htbp]
  \centering
  \includegraphics[width=0.95\linewidth]{figures/exp2_jan.png}
  \caption{実験2の結果(1月):importanceとscore impactの整合性の可視化}
  \label{fig:exp2_jan}
\end{figure}

\begin{figure}[htbp]
  \centering
  \includegraphics[width=0.95\linewidth]{figures/exp2_feb.png}
  \caption{実験2の結果(2月):importanceとscore impactの整合性の可視化}
  \label{fig:exp2_feb}
\end{figure}

\begin{figure}[htbp]
  \centering
  \includegraphics[width=0.95\linewidth]{figures/exp2_mar.png}
  \caption{実験2の結果(3月):importanceとscore impactの整合性の可視化}
  \label{fig:exp2_mar}
\end{figure}


\section{まとめ}
本章では,学習済みモデルを固定したまま実行可能な評価として,
(1) ランキング指標による予測性能の確認,
(2) 実験1としてマスク値(importance)の分布確認(1月,2月,3月),
(3) 実験2として重要度と実影響(score impact)の整合性チェック(1月,2月,3月),
を実施し,説明が利用可能かどうかを検証するための結果を整理した.
次章では,これらの結果を踏まえて考察と限界,および今後の展望を述べる.
