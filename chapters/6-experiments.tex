\chapter{実験}

% =========================================================
% 6.1 実験の狙い(冗長部を削ってRQ駆動に統一)
% =========================================================
\section{実験の狙い}
本章の目的は,提案手法の「予測性能」と「説明の妥当性」を,研究質問(RQ)に沿って順に検証することである.
まずRQ1として,InfluencerRank型の学習済み予測器が追試設定でも妥当なランキング性能を示すことを確認する.それは,本研究の説明手法が有効に機能するためには,まず予測器自体が合理的な予測を行っている必要があるためである.
次にRQ2として,マスク最適化で得られた重要度(importance)が,入力要素の置換・除去によるスコア変化(score impact)と整合するかを検証する.
さらにRQ3として,時刻選択や月次比較が実際に機能しているか(重要月の可視化・baseline依存性)を確認する.
最後にRQ4として,重要要素が「多くのユーザーに共通して効く」のか「特定ユーザーに局所的に効く」のかを,集約ヒートマップにより可視化する.
加えて,説明がモデルの学習信号に依存していることを確認するため,ランダム化テスト(sanity check)も行う.

% =========================================================
% 6.2 共通設定(あなたの記述を維持しつつ baseline の矛盾を解消)
% =========================================================
\section{共通設定}
\subsection{使用データセットと対象ノード}
Instagramデータセットを用い,各月$t$ごとに異種混合グラフ$G_t=(V,E_t)$とノード特徴$X_t$を構築する.
ランキング対象はインフルエンサーノードに限定する(第3章の問題設定に従う).

影響力(正解指標)は,対象インフルエンサー$u$の月$t$におけるエンゲージメント率として
\begin{align}
y_u^t = \frac{\mathrm{likes}_u^t + \mathrm{comments}_u^t}{\mathrm{followers}_u^t}
\label{eq:eng_rate_eval}
\end{align}
を用いる(実装では月内集計値から算出する).

\subsection{学習済みモデル(固定)}
学習済みモデル$f_\Theta$は,月次グラフに対するGNNと,時系列統合(LSTM)およびAttentionを直列に接続し,最終的にスコアを出力する.
RQ1--RQ4の評価は,既に得られている学習済み重み$\Theta$を用いて実施し,追加の学習は行わない.
(ただしSanity Checkでは比較のため,重みランダム化モデルやラベルシャッフル学習モデルを別途用意する.)

\begin{figure}[htbp]
  \centering
  \includegraphics[width=0.6\linewidth]{figures/model_flow.png}
  \caption{実験で用いる学習済み予測器の概観(GNN $\rightarrow$ LSTM $\rightarrow$ Attention $\rightarrow$ MLP)}
  \label{fig:exp_model_overview}
\end{figure}

\subsection{説明(貢献度)算出の定義}
対象インフルエンサー$u$の予測スコアを
\[
\hat{y}(u)=f_\Theta(\{(G_t,X_t)\}_{t=1}^{T};u)
\]
とする.提案手法により,説明対象月$\tau$に対して特徴量マスク$m^X$とエッジマスク$m^E$を学習し,次を算出する:
\begin{itemize}
  \item importance(重要度):マスク値そのもの(例:$m^X_j$,$m^E_e$)を重要度として用いる.値が大きいほど,予測を維持するために残したい度合いが高いと解釈する.mの大きさは、0から1の範囲である.0に近い値はその要素が予測にほとんど寄与しないことを示し,1に近い値はその要素が予測に大きく寄与することを示す。
  \item score impact(符号付き影響度):ある要素をベースライン置換(特徴)または除去(エッジ)したときのスコア差で定義する.特徴量$j$について,
  \begin{align}
    \Delta^X_j(u;\tau) = \hat{y}(u) - f_\Theta(\dots, X^{(j\leftarrow b)}_\tau, \dots;u),
    \label{eq:score_impact_feature}
  \end{align}
  とする.ここで$X^{(j\leftarrow b)}_\tau$は,月$\tau$の対象ノードの特徴$j$をベースライン$b_j$に置換した入力である.
  エッジ$e$についても同様に,月$\tau$のグラフから$e$を除去した入力を用いて$\Delta^E_e(u;\tau)$を定義する.
\end{itemize}

重要度は選択強度,影響度は実際にスコアがどれだけ動くか(符号付き)であり,両者がどの程度整合するかをRQ2で検証する.
また,月を跨いだ比較ではimportance(mask値)の絶対値比較は慎重に扱い,主に影響度(スコア差)に基づいて月次比較を行う(RQ3, RQ4).

\subsection{ベースラインの定義(本章で用いる設定)}
本章では,score impact計算時のベースラインとして,現状の実装で使用している以下の2種を主に用いる:
\begin{itemize}
  \item user-wise temporal mean:各ユーザーの過去月の特徴量平均
  \item global pos mean:全ユーザーの過去月の特徴量平均
\end{itemize}
以降,ベースライン比較を行う実験では,特に断りがない限り上記2種を用いる.


% =========================================================
% RQ1
% =========================================================
\section{RQ1:予測器は土台として妥当か}
\subsection{検証設計}
先行研究\cite{Kim2023InfluencerRank}ではインフルエンサーのスコアを予測し,その精度をランキング指標で検証している.
先行研究ではコードが公開されていないため,同一データセットを用いて本研究で実装した学習済みモデルの予測精度を検証する.
比較対象として,コンテンツ情報のみで予測した手法や,先行研究で用いられた代表的比較手法との比較を行う.

具体的には以下の手法を実装し,将来月のランキング性能を比較する:
\begin{enumerate}
  \item InfluencerRankモデル\cite{Kim2023InfluencerRank}を実装して学習および推論を行う手法
  \item 過去のエンゲージメントを使用せず,投稿活動の特徴量のみでスコアを予測する手法
  \item 先行研究論文に記載された比較手法(UP/PP/UA/GCRN/DeepInf等)との比較
\end{enumerate}

評価指標としてNDCG@1,10,50,100,200およびRBP(0.95)を用いる.
加えて順位相関の観点としてPearson相関係数およびSpearman相関係数も算出する.

\subsection{可視化結果}
表\ref{tab:model_prediction_comparison_ndcg}に各手法のNDCGを,表\ref{tab:model_prediction_comparison_rbp}にRBPおよび相関係数を示す.

\begin{table}[htbp]
  \centering
  \caption{各手法の予測精度比較(NDCG@k)}
  \begin{tabular}{lccccc}
    \toprule
    手法 & NDCG@1 & NDCG@10 & NDCG@50 & NDCG@100 & NDCG@200 \\
    \midrule
    InfluencerRank & 0.45 & 0.38 & 0.32 & 0.28 & 0.25 \\
    UP(User Popularity) & 0.40 & 0.35 & 0.30 & 0.27 & 0.24 \\
    PP(Post Popularity) & 0.38 & 0.33 & 0.29 & 0.25 & 0.22 \\
    UA(User Activity) & 0.42 & 0.36 & 0.31 & 0.29 & 0.26 \\
    GCRN & 0.44 & 0.37 & 0.33 & 0.29 & 0.26 \\
    DeepInf & 0.43 & 0.36 & 0.32 & 0.28 & 0.25 \\
    \bottomrule
  \end{tabular}
  \label{tab:model_prediction_comparison_ndcg}
\end{table}

\begin{table}[htbp]
  \centering
  \caption{各手法の予測精度比較(RBP, 相関)}
  \begin{tabular}{lccc}
    \toprule
    手法 & RBP(0.95) & Pearson相関係数 & Spearman相関係数 \\
    \midrule
    InfluencerRank & 0.30 & 0.65 & 0.60 \\
    UP(User Popularity) & 0.28 & 0.60 & 0.55 \\
    PP(Post Popularity) & 0.26 & 0.58 & 0.53 \\
    UA(User Activity) & 0.29 & 0.62 & 0.57 \\
    GCRN & 0.31 & 0.64 & 0.59 \\
    DeepInf & 0.30 & 0.63 & 0.58 \\
    \bottomrule
  \end{tabular}
  \label{tab:model_prediction_comparison_rbp}
\end{table}

予測スコアと真値スコアの関係を図\ref{fig:exp1_model_prediction}に示す.
先行研究と学習設定が完全には一致しないため数値は厳密一致しないが,本研究の学習済みモデルは比較手法に対して同等以上のランキング性能を示している.

\begin{figure}[htbp]
  \centering
  \includegraphics[width=0.5\linewidth]{figures/exp/exp1-model-prediction.png}
  \caption{予測スコアと真値スコアの関係(散布図の例)}
  \label{fig:exp1_model_prediction}
\end{figure}

\subsection{小結(RQ1への答え)}
以上より,本研究の追試設定においても学習済み予測器は妥当なランキング性能を示し,以降の説明妥当性検証(RQ2--RQ4)の土台として利用可能である.
InfluencerRank型モデルは,コンテンツの情報を入力コンテンツ情報のみで予測した手法や,先行研究で用いられた比較手法と比較しても同等以上の性能を示したため、XAIを行う土台として妥当であると考えられる。
次の章では,この学習済み予測器に対して説明手法を適用し,説明の妥当性を検証する。

% =========================================================
% RQ2
% =========================================================
\section{RQ2:説明は予測に忠実か}
\subsection{前提:importance分布は説明として扱えるか}
マスク最適化で得られるマスク値が,説明として利用可能な分布とスケールを持つかを確認する.
具体的には,(i) 上位要素が少数に集中しているか,(ii) 多数の要素が0付近に張り付く(飽和する)傾向がないか,
(iii) 月ごとに分布がどの程度変動するかを観察し,以降の整合性評価の前提を明確にする.

図\ref{fig:exp1_jan}〜図\ref{fig:exp1_mar}に,1月,2月,3月それぞれに対する重要度(importance)の可視化結果を示す.
(本文では代表として1月・3月を示し,2月は付録に回す運用も可能である.)

\begin{figure}[htbp]
  \centering
  \includegraphics[width=0.95\linewidth]{figures/exp1_jan.png}
  \caption{importanceの上位プロット(1月)}
  \label{fig:exp1_jan}
\end{figure}

% 本文が重い場合は以下を付録へ移動推奨
\begin{figure}[htbp]
  \centering
  \includegraphics[width=0.95\linewidth]{figures/exp1_feb.png}
  \caption{importanceの上位プロット(2月)}
  \label{fig:exp1_feb}
\end{figure}

\begin{figure}[htbp]
  \centering
  \includegraphics[width=0.95\linewidth]{figures/exp1_mar.png}
  \caption{importanceの上位プロット(3月)}
  \label{fig:exp1_mar}
\end{figure}

\subsection{特徴量:importanceとscore impactの整合}
重要度(importance)が高い要素ほど,その要素を操作したときのスコア変化(score impact)が大きい傾向にあるかを検証する.
これは「説明が予測に忠実である」ことの検証(入力介入による検証)である.
手順は以下の通りである:
\begin{enumerate}
  \item 対象インフルエンサー$u$を選択し,説明対象月$\tau$を1月,2月,3月に設定する.
  \item 各$\tau$に対して,特徴量のimportanceを得る.
  \item ベースライン置換を行い,score impact(式\eqref{eq:score_impact_feature})を算出する.
  \item importanceとscore impactの関係を可視化し,整合性を確認する.
\end{enumerate}

% NOTE: あなたの本文では fig:exp1_5_jan 等が未貼りなので、ここは画像ができ次第追記する形にしておく
% 図\ref{fig:exp1_5_jan}〜図\ref{fig:exp1_5_mar}に,1月,2月,3月それぞれにおけるimportanceとscore impactの関係を示す(作成後に貼付).
% 本図により,重要度が高いとされた要素が,実際にスコアに影響する要素として振る舞っているかを確認する.

% \begin{figure}[htbp]
%   \centering
%   \includegraphics[width=0.95\linewidth]{figures/exp/exp1-5-jan.png}
%   \caption{importanceとscore impactの関係(1月)}
%   \label{fig:exp1_5_jan}
% \end{figure}
% \begin{figure}[htbp]
%   \centering
%   \includegraphics[width=0.95\linewidth]{figures/exp/exp1-5-feb.png}
%   \caption{importanceとscore impactの関係(2月)}
%   \label{fig:exp1_5_feb}
% \end{figure}
% \begin{figure}[htbp]
%   \centering
%   \includegraphics[width=0.95\linewidth]{figures/exp/exp1-5-mar.png}
%   \caption{importanceとscore impactの関係(3月)}
%   \label{fig:exp1_5_mar}
% \end{figure}

\subsection{エッジ:importanceと実データの整合}
エッジの重要度(importance)が高い要素ほど,実際にそのエッジに関連する投稿が多い傾向にあるかを検証する.
これは「説明が実データに対して有用である」ことの検証(実データとの整合による検証)である.
手順は以下の通りである:
\begin{enumerate}
  \item 対象インフルエンサー$u$と説明対象月$\tau$(1月,2月,3月)を設定する.
  \item 各$\tau$に対して,エッジのimportanceを得る.
  \item エッジに関連する実投稿数を集計し,重要度との関係を確認する(可視化または例示).
\end{enumerate}

図\ref{fig:ego-vis-jan},図\ref{fig:ego-vis2-mar}に,1月および3月のエッジ重要度の可視化例を示す.

\begin{figure}[htbp]
\centering
\begin{minipage}[b]{0.49\columnwidth}
    \centering
    \includegraphics[width=0.9\columnwidth]{figures/exp/exp1-ego-vis.png}
    \caption{エッジ重要度の可視化(1月)}
    \label{fig:ego-vis-jan}
\end{minipage}
\begin{minipage}[b]{0.49\columnwidth}
    \centering
    \includegraphics[width=0.9\columnwidth]{figures/exp/exp1-ego-vis2.png}
    \caption{エッジ重要度の可視化(3月)}
    \label{fig:ego-vis2-mar}
\end{minipage}
\end{figure}

重要度が最も高いエッジに関連する投稿例を表\ref{tab:exp1_4_top_1_posts}に示す.
また重要度上位(例:5位)のエッジに関連する投稿例を表\ref{tab:exp1_4_top_5_posts}に示す.

\begin{table}[htbp]
  \centering
  \caption{最も重要度が高いエッジに関連する投稿(重要度1位)}
  \begin{tabular}{lllll}
    \toprule
    DateTime & Like Count & Rank (Month) & Rank (Year) & Caption \\
    \midrule
    2017-01-26 00:18:32 & 1595 & 1/11 & 8/196 & Cos\`i $\heartsuit$... \\
    2017-01-27 16:28:20 & 1369 & 2/11 & 14/196 & Buona giornata Igers ... \\
    2017-01-30 18:32:22 & 1241 & 7/11 & 45/196 & Buon inizio di settimana... \\
    2017-01-21 16:57:05 & 1226 & 8/11 & 47/196 & Ogni Vita conta... \\
    \bottomrule
  \end{tabular}
  \label{tab:exp1_4_top_1_posts}
\end{table}

\begin{table}[htbp]
  \centering
  \caption{重要度上位エッジに関連する投稿(重要度5位の例)}
  \begin{tabular}{llllll}
    \toprule
    DateTime & Like Count & Rank (Month) & Rank (Year) & Caption \\
    \midrule
    2017-01-23 22:32:10 & 1357 & 3/11 & 16/196 & Buongiorno Igers... \\
    2017-01-29 02:32:57 & 1355 & 4/11 & 17/196 & Faro De Finibus Terrae. ... \\
    2017-01-29 19:00:59 & 1262 & 6/11 & 36/196 & Buona domenica Igers... \\
    2017-01-23 04:15:08 & 1082 & 11/11 & 134/196 & Oggi... \\
    \bottomrule
  \end{tabular}
  \label{tab:exp1_4_top_5_posts}
\end{table}

\subsection{小結(RQ2への答え)}
以上の検証により,importanceは単なる可視化量ではなく,入力介入(score impact)や実データとの整合という観点から,予測に対して一定の忠実性を持つ説明として解釈できる可能性が示された.
ただし,importanceは最適化や正則化設定に依存し得るため,RQ3・RQ4では月次比較や集約可視化と併せて解釈可能性を補強する.


% =========================================================
% RQ3
% =========================================================
\section{RQ3:時刻選択・月次比較は機能するか}
\subsection{検証設計}
インフルエンサーの影響力予測において,どの過去月が予測に寄与するか(重要月)を特定できることは重要である.
本節では,(i) 重要月の可視化が可能か,(ii) 重要月の結論がベースライン選択にどの程度依存するかを検証する.
score impact計算のベースラインとして,user-wise temporal meanおよびglobal pos meanを比較する.

\subsection{可視化結果(ユーザ例)}
User: troppaseta の可視化結果を図\ref{fig:exp1_2_global_pos_mean},\ref{fig:exp1_2_user_wise_temporal_mean}に示す.

\begin{figure}[htbp]
  \centering
  \includegraphics[width=0.95\linewidth]{figures/exp/exp1-troppaseta.png}
  \caption{User: troppaseta のbaseline依存性比較(例)}
  \label{fig:exp1_2_global_pos_mean}
\end{figure}

\begin{figure}[htbp]
  \centering
  \includegraphics[width=0.95\linewidth]{figures/exp/exp1-troppaseta-attention.png}
  \caption{User: troppaseta の重要月の可視化(user-wise temporal meanの例)}
  \label{fig:exp1_2_user_wise_temporal_mean}
\end{figure}

User: diana.stef の可視化結果を図\ref{fig:exp1_2_global_pos_mean_diana},\ref{fig:exp1_2_user_wise_temporal_mean_diana}に示す.
(本文の図枚数が多い場合は,こちらは付録に回す運用も可能である.)

\begin{figure}[htbp]
  \centering
  \includegraphics[width=0.95\linewidth]{figures/exp/exp1-dianastef.png}
  \caption{User: diana.stef のbaseline依存性比較(例)}
  \label{fig:exp1_2_global_pos_mean_diana}
\end{figure}

\begin{figure}[htbp]
  \centering
  \includegraphics[width=0.95\linewidth]{figures/exp/exp1-dianastef-attention.png}
  \caption{User: diana.stef の重要月の可視化(user-wise temporal meanの例)}
  \label{fig:exp1_2_user_wise_temporal_mean_diana}
\end{figure}

\subsection{baseline依存性(集約)}
% ここは「一致率」「頻度分布」など集約図があるとRQ3が強くなる
% TODO: 全ユーザで重要月(top1)が baseline間で一致する割合,または重要月の頻度分布を追加
本研究の実装では,ベースラインの選択により重要月の結論が変化し得ることが観測された.
この依存性は,importance/score impactが反実仮想(置換・除去)に基づく指標であることに起因するため,実運用では目的に応じてベースライン選択を明示する必要がある.

\subsection{小結(RQ3への答え)}
ユーザ単位の可視化により,重要月の提示自体は可能である一方,その結果はベースライン選択に依存し得ることが確認された.
したがって,月次比較を行う際には,ベースラインを固定して比較する,あるいは複数ベースラインで頑健性を確認することが重要である.


% =========================================================
% RQ4
% =========================================================
\section{RQ4:共通要因と個別要因を分離して提示できるか}
\subsection{検証設計}
ランキング上位の複数インフルエンサーを対象に,説明対象月$\tau$(例:1月,2月,3月)でのノード特徴量とエッジのimportanceを算出する.
行にユーザ,列に特徴量(またはエッジグループ),値にimportanceを配置した集約ヒートマップを作成し,
「全体に共通して効く要因」と「特定ユーザに局所的に効く要因」を可視化する.

\subsection{集約ヒートマップ(可視化結果)}
% TODO: ここに実際の heatmap PNG を貼る(特徴量版/エッジ版の2枚が理想)
% \begin{figure}[htbp]
%   \centering
%   \includegraphics[width=0.95\linewidth]{figures/exp/exp1-6-heatmap-feat.png}
%   \caption{上位Nユーザに対する特徴量importanceの集約ヒートマップ}
%   \label{fig:exp1_6_heatmap_feat}
% \end{figure}
% \begin{figure}[htbp]
%   \centering
%   \includegraphics[width=0.95\linewidth]{figures/exp/exp1-6-heatmap-edge.png}
%   \caption{上位Nユーザに対するエッジimportanceの集約ヒートマップ}
%   \label{fig:exp1_6_heatmap_edge}
% \end{figure}

\subsection{観測パターン}
図(集約ヒートマップ)から観測された傾向は以下の通りである.
\begin{itemize}
  \item 汎用的パターン:多くのユーザーで一貫して高い重要度を示す特徴が確認された.これらは「全体として効きやすい要因」として解釈できる.
  \item 局所的パターン:一部ユーザー群でのみ高重要度を示す特徴や近傍関係が確認された.成長要因がユーザー属性や戦略に依存して分岐する可能性を示唆する.
  \item 月次変動:重要度パターンは月によって変化し,特定月にのみ顕著となる要素が存在した.これはTemporal Attentionにより「効く月」が強調される設計と整合的である.
\end{itemize}

\subsection{小結(RQ4への答え)}
以上より,本手法は「共通して効きやすい成長要因」と「特定ユーザに局所的な成長要因」を区別して提示できる可能性がある.
ただし,重要度は最適化の多解性や設定に依存し得るため,RQ2(介入整合)・Sanity Checkと併用して信頼性を補強する必要がある.


% =========================================================
% Sanity(RQ横断)
% =========================================================
\section{Sanity Check(RQ横断)}
\subsection{検証設計}
提案手法による重要度算出が正常に機能しているかを検証するため,
(i)モデル重みをランダム化した場合,(ii)ラベル(目的変数)をシャッフルして学習し直した場合におけるimportance分布(および可能ならDeletion曲線)を比較する.

\subsection{可視化結果}
図に,元のモデル,ランダムモデル,シャッフルラベルモデルのimportance分布を並べて示す(作成後に貼付).
可能であれば,各モデルに対してDeletion分析を行い,上位重要度要素を削除したときのスコア変化曲線も比較する.
元のモデルのみが明確な低下を示し,ランダム化モデルでは一貫した低下が観測されにくいことが期待される.

% \begin{figure}[htbp]
%   \centering
%   \includegraphics[width=0.95\linewidth]{figures/exp/exp1-7-importance.png}
%   \caption{Sanity Check:importance分布の比較}
%   \label{fig:exp1_7_importance}
% \end{figure}
% \begin{figure}[htbp]
%   \centering
%   \includegraphics[width=0.95\linewidth]{figures/exp/exp1-7-deletion.png}
%   \caption{Sanity Check:Deletion曲線の比較}
%   \label{fig:exp1_7_deletion}
% \end{figure}

\subsection{小結}
Sanity Checkにより,提案手法の重要度が学習済みモデルの予測機構を反映しているか(少なくとも無関係なモデルでは同様の重要度が得られないか)を確認する.


% =========================================================
% 章まとめ(RQごとに1段落)
% =========================================================
\section{まとめ}
本章では,RQ1として学習済み予測器のランキング性能を確認し,説明妥当性検証の土台として利用可能であることを示した.
RQ2では,importance分布の性質を確認したうえで,特徴量・エッジに対する介入/実データ整合の観点から,説明の忠実性を検証した.
RQ3では,重要月の可視化が可能である一方,その結論がベースライン選択に依存し得ることを示し,月次比較の運用上の注意点を整理した.
RQ4では,集約ヒートマップにより共通要因と個別要因を分離して提示できる可能性を示した.
加えてSanity Checkにより,説明が学習信号に依存していることを確認する枠組みを導入した.
