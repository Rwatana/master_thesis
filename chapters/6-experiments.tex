\chapter{実験}

\section{実験の目的}
提案手法としてマスクを使用したXAIを導入しているが,その説明が妥当な振る舞いをするかを検証する必要がある.
本研究では,先行研究をフルスクラッチで実装し学習、推論したものに対して,最適化ベースのマスク学習により得られる説明が妥当な振る舞いをするかを検証する.
ここで妥当性とは,重要とされた要素(特徴量・エッジ)を操作すると予測が想定通りに変化すること,および説明が少数要素にまとまり月次で比較可能な形で提示できることを指す.
また、その説明が実データに対して有用であるかも検証する.

本章の目的は以下の2点である.
\begin{enumerate}
  \item 予測性能の定量評価(学習済みモデルの確認):
  学習済みモデルが,将来月の影響力ランキングをどの程度整合的に出力できているかを,ランキング指標(NDCG,RBP 等)で評価する.
  \item 説明(貢献度)の妥当性検証(学習なしで可能な検証):
  マスク最適化で得られた重要度(importance)と,アブレーションで得る符号付き影響度(score impact)が,予測スコアの変化と整合するかを検証する.
\end{enumerate}

その上で、まず大きく二段階で実験を行う.
\begin{itemize}
  \item 一段階目は,学習済みモデルの予測性能をランキング指標で評価する.
  \item 二段階目は,再学習を行い重要な特徴がなくなると予測性能がどの程度低下するかを評価する.
\end{itemize}



\section{実験設定}
\subsection{使用データセットと対象ノード}
Instagram データセットを用い,各月 $t$ ごとに異種混合グラフ $G_t=(V,E_t)$ とノード特徴 $X_t$ を構築する.
ランキング対象はインフルエンサーノードに限定する(第3章の問題設定に従う).

影響力(正解指標)は,対象インフルエンサー $u$ の月 $t$ におけるエンゲージメント率として
\begin{align}
y_u^t = \frac{\mathrm{likes}_u^t + \mathrm{comments}_u^t}{\mathrm{followers}_u^t}
\label{eq:eng_rate_eval}
\end{align}
を用いる(実装では月内集計値から算出する).

\subsection{学習済みモデル(固定)}
学習済みモデル $f_\Theta$ は,月次グラフに対するGNNと,時系列統合(LSTM)およびAttentionを直列に接続し,最終的にスコアを出力する.
本章では,既に得られている学習済み重み $\Theta$ を用い,追加の学習は行わない.

\begin{figure}[htbp]
  \centering
  \includegraphics[width=0.85\linewidth]{figures/lstm-attention-mlp.png}
  \caption{実験で用いる学習済み予測器の概観(GNN $\rightarrow$ LSTM $\rightarrow$ Attention $\rightarrow$ MLP)}
  \label{fig:exp_model_overview}
\end{figure}

\subsection{説明(貢献度)算出の定義}
対象インフルエンサー $u$ の予測スコアを
\[
\hat{y}(u)=f_\Theta(\{(G_t,X_t)\}_{t=1}^{T};u)
\]
とする.提案手法により,説明対象月 $\tau$ に対して特徴量マスク $m^X$ とエッジマスク $m^E$ を学習し,次を算出する:
\begin{itemize}
  \item importance(重要度):マスク値そのもの(例:$m^X_j$,$m^E_e$)を重要度として用いる.値が大きいほど,予測を維持するために残したい度合いが高いと解釈する.
  \item score impact(符号付き影響度):ある要素をベースライン置換(特徴)または除去(エッジ)したときのスコア差で定義する.特徴量 $j$ について,
  \begin{align}
    \Delta^X_j(u;\tau) = \hat{y}(u) - f_\Theta(\dots, X^{(j\leftarrow b)}_\tau, \dots;u),
    \label{eq:score_impact_feature}
  \end{align}
  とする.ここで $X^{(j\leftarrow b)}_\tau$ は,月 $\tau$ の対象ノードの特徴 $j$ をベースライン $b_j$ に置換した入力である.
  エッジ $e$ についても同様に,月 $\tau$ のグラフから $e$ を除去した入力を用いて $\Delta^E_e(u;\tau)$ を定義する.
\end{itemize}

重要度は選択強度,影響度は実際にスコアがどれだけ動くか(符号付き)であり,両者がどの程度整合するかを検証する.
また,月を跨いだ比較では重要度(mask値)の絶対値比較は慎重に扱い,主に影響度(スコア差)に基づいて月次比較を行う.


\section{説明の妥当性検証:学習なしで可能な検証}

\section {実験1.1 学習済みモデルの予測精度の比較}
\subsection{目的}
先行研究\cite{Kim2023InfluencerRank}ではインフルエンサーのスコアを予測してそのスコアに対する精度を検証していた。先行研究ではコードが公開されていないため、同じデータセットを用いて本研究で使用している学習済みモデルの予測精度を検証する。
比較対象として、コンテンツの力のみで予測したものと比較する。また、先行研究で使用されていたモデルも使用して比較する。

\subsection{手順}
以下のそれぞれの手法で実装を行い、予測精度の比較を行う。
\begin{enumerate}
  \item Instagram データセットを用いて、Influencer rankモデル\cite{Kim2023InfluencerRank}を実装して学習及び推論を行う手法
  \item 過去のエンゲージメントの値を使用しないで、投稿活動の特徴量のみを使用して各インフルエンサーのスコアを予測する手法
  \item 先行研究の論文で使用されていた比較手法と精度の比較を行う。
\end{enumerate}

評価指標としては先行研究で使用されていた、NDCG@1, NDCG@10, NDCG@50, NDCG@100, NDCG@200, RBP(0.95)を使用する。
新たに順位としての相関を示すpearson相関係数, spearman相関係数も追加で評価指標として使用する。

\subsection{結果}
予測スコアとTrueスコアの関係を図\ref{fig:exp1_model_prediction}に示す。

\begin{figure}[htbp]
  \centering
  \includegraphics[width=0.95\linewidth]{figures/exp/exp1-model-prediction.png}
  \caption{実験1の結果(1月):ノード特徴量とエッジのimportanceの上位プロット}
  \label{fig:exp1_model_prediction}
\end{figure}

表\ref{tab:model_prediction_comparison_ndcg}に各手法の予測精度を示す。
% 表を作成して、各手法の予測精度を示す(NDCG)
\begin{table}[htbp]
  \centering
  \caption{各手法の予測精度比較}
  \begin{tabular}{lccccc}
    \toprule
    手法 & NDCG@1 & NDCG@10 & NDCG@50 & NDCG@100 & NDCG@200 \\
    \midrule
    InfluencerRank & 0.45 & 0.38 & 0.32 & 0.28 & 0.25 \\
    UP(User Popularity) & 0.40 & 0.35 & 0.30 & 0.27 & 0.24 \\
    PP(Post Popularity) & 0.38 & 0.33 & 0.29 & 0.25 & 0.22 \\
    UA(User Activity) & 0.42 & 0.36 & 0.31 & 0.29 & 0.26 \\
    GCRN & 0.44 & 0.37 & 0.33 & 0.29 & 0.26 \\
    DeepInf & 0.43 & 0.36 & 0.32 & 0.28 & 0.25 \\
    \bottomrule
  \end{tabular}
  \label{tab:model_prediction_comparison_ndcg}
\end{table} 

追加で順位としての相関を示すpearson相関係数, spearman相関係数も含めた予測精度を表\ref{tab:model_prediction_comparison_rbp}に示す。
% 表を作成して書く手法の予測精度を示す(RBP  ,Pearson, Spearman)
\begin{table}[htbp]
  \centering
  \caption{各手法の予測精度比較}
  \begin{tabular}{lccc}
    \toprule
    手法 & RBP(0.95) & Pearson相関係数 & Spearman相関係数 \\
    \midrule
    InfluencerRank & 0.30 & 0.65 & 0.60 \\
    UP(User Popularity) & 0.28 & 0.60 & 0.55 \\
    PP(Post Popularity) & 0.26 & 0.58 & 0.53 \\
    UA(User Activity) & 0.29 & 0.62 & 0.57 \\
    GCRN & 0.31 & 0.64 & 0.59 \\
    DeepInf & 0.30 & 0.63 & 0.58 \\
    \bottomrule
  \end{tabular}
  \label{tab:model_prediction_comparison_rbp}
\end{table}

\section{実験1.2 重要な月の特定}
\subsection{目的}
本研究では,インフルエンサーの影響力予測において特に重要な月を特定することが重要である。そのために、過去のデータを分析し、影響力の変動が大きい月を特定する。
具体的には、各インフルエンサーの月次エンゲージメント率の変動を計算し、変動が大きい月を重要な月として特定する。この分析により、説明手法の評価において注目すべき月を明確にする。
\subsection{手順}
重要だと考えられる月を特定する手段として以下の手法それぞれが考えられる。
\begin{enumerate}
  \item 本研究では,pos 重要度評価における baseline 依存性を検証するため,以下の 4 種類の baseline を用いて比較実験を行った:
  \begin{enumerate}
    \item user-wise temporal mean(各ユーザーの過去月の特徴量の平均)
    \item global pos mean(全ユーザーの過去月の特徴量の平均)
  \end{enumerate}
\end{enumerate}

各ユーザーの特徴量の平均に関しては、そのユーザーの過去月の特徴量の平均を計算し、baseline として使用する。
全ユーザーの特徴量の平均に関しては、全てのユーザーの過去月の特徴量の平均を計算し、baseline として使用する。
その狙いとしては、user-wise temporal mean は各ユーザーの特徴量の変動を考慮する一方で、global pos mean は全体的な傾向を捉えることができるため、baseline の選択が説明手法の評価に与える影響を比較することができる。

\subsection{結果}

User: troppaseta の可視化結果を示す
\begin{figure}[htbp]
  \centering
  \includegraphics[width=0.95\linewidth]{figures/exp/exp1-troppaseta.png}
  \caption{実験1.2の結果:User: troppaseta のbaseline依存性の比較}
  \label{fig:exp1_2_global_pos_mean}
\end{figure}

\begin{figure}[htbp]
  \centering
  \includegraphics[width=0.95\linewidth]{figures/exp/exp1-troppaseta-attention.png}
  \caption{実験1.2の結果:user-wise temporal mean を baseline とした場合の重要な月の特定}
  \label{fig:exp1_2_user_wise_temporal_mean}
\end{figure}


User diana.stef の可視化結果を示す
\begin{figure}[htbp]
  \centering
  \includegraphics[width=0.95\linewidth]{figures/exp/exp1-dianastef.png}
  \caption{実験1.2の結果:User: diana.stef のbaseline依存性の比較}
  \label{fig:exp1_2_global_pos_mean_diana}
\end{figure}

\begin{figure}[htbp]
  \centering
  \includegraphics[width=0.95\linewidth]{figures/exp/exp1-dianastef-attention.png}
  \caption{実験1.2の結果:user-wise temporal mean を baseline とした場合の重要な月の特定}
  \label{fig:exp1_2_user_wise_temporal_mean_diana}
\end{figure}

\section{実験1.3:マスク値の確認とスコアへの影響度}
\subsection{目的}
マスク最適化で得られるマスク値が,説明として利用可能な分布とスケールを持つかを確認する.
具体的には,(i) 上位要素が少数に集中しているか,(ii) 多数の要素が0付近に張り付く(飽和する)傾向がないか,
(iii) 月ごとに分布がどの程度変動するかを観察し,以降の整合性評価の前提を明確にする.


\subsection{手順}
\begin{enumerate}
  \item 対象インフルエンサー $u$ を選択する(例:将来月の予測ランキング上位).
  \item 説明対象月 $\tau$ を 1月,2月,3月に設定し,各 $\tau$ に対して特徴量マスク $m^X$ とエッジマスク $m^E$ を得る.
  \item 各マスクを降順ソートし,top要素を抽出して可視化する(ノード特徴量とエッジを別々に表示する).
\end{enumerate}

\subsection{結果}
図\ref{fig:exp1_jan}〜図\ref{fig:exp1_mar}に,1月,2月,3月それぞれに対する重要度(importance)の可視化結果を示す.
これらの図は,マスク最適化により得られた重要度を上位からプロットし,値の分布がどの程度広がるかを確認する目的で作成したものである.

\begin{figure}[htbp]
  \centering
  \includegraphics[width=0.95\linewidth]{figures/exp1_jan.png}
  \caption{実験1の結果(1月):ノード特徴量とエッジのimportanceの上位プロット}
  \label{fig:exp1_jan}
\end{figure}

\begin{figure}[htbp]
  \centering
  \includegraphics[width=0.95\linewidth]{figures/exp1_feb.png}
  \caption{実験1の結果(2月):ノード特徴量とエッジのimportanceの上位プロット}
  \label{fig:exp1_feb}
\end{figure}

\begin{figure}[htbp]
  \centering
  \includegraphics[width=0.95\linewidth]{figures/exp1_mar.png}
  \caption{実験1の結果(3月):ノード特徴量とエッジのimportanceの上位プロット}
  \label{fig:exp1_mar}
\end{figure}


\section{実験1.4: エッジの重要度と実投稿の相関}
\subsection{目的}
エッジの重要度(importance)が高い要素ほど,実際にそのエッジに関連する投稿が多い傾向にあるかを検証する.
これは「説明が実データに対して有用である」ことの,学習不要な検証(実データとの相関による検証)である.
\subsection{手順}
\begin{enumerate}
  \item 実験1と同一の対象インフルエンサー $u$ を用い,説明対象月 $\tau$ を 1月,2月,3月に設定する.
  \item 各 $\tau$ に対して,エッジのimportanceを得る.
  \item エッジに関連する実投稿数を集計する.
  \item importanceと実投稿数の関係を可視化し,整合性を確認する.
\end{enumerate}

\subsection{結果}
エッジの重要度は以下のように得られる。
% グラフのエッジ重要度
\begin{figure}[htbp]
  \centering
  \includegraphics[width=0.95\linewidth]{figures/exp/exp1-ego-vis.png}
  \caption{実験1の結果(1月):importanceと実投稿数の関係の可視化}
  \label{fig:exp1_4_jan}
\end{figure}


また、その投稿の重要度が一番高いもののを含んだ投稿は以下のような表になる。

\begin{table}[htbp]
  \centering
  \caption{最も重要度が高いエッジに関連する投稿(重要度1位)}
  \begin{tabular}{lllll}
    \toprule
    DateTime & Like Count & Rank (Month) & Rank (Year) & Caption \\
    \midrule
    2017-01-26 00:18:32 & 1595 & 1/11 & 8/196 & Così $\heartsuit$... \\
    2017-01-27 16:28:20 & 1369 & 2/11 & 14/196 & Buona giornata Igers \textcolor{blue}{\large$\blacktriangle$}... \\
    2017-01-30 18:32:22 & 1241 & 7/11 & 45/196 & Buon inizio di settimana... \\
    2017-01-21 16:57:05 & 1226 & 8/11 & 47/196 & Ogni Vita conta... \\
    \bottomrule
  \end{tabular}
  \label{tab:exp1_4_top_1_posts}
\end{table}

重要度5位に関して行うと、
\begin{table}[htbp]
  \centering
  \caption{最も重要度が高いエッジに関連する投稿(重要度5位)}
  \begin{tabular}{llllll}
    \toprule
    DateTime & Like Count & Rank (Month) & Rank (Year) & Caption \\
    \midrule
    2017-01-23 22:32:10 & 1357 & 3/11 & 16/196 & Buongiorno Igers... \\
    2017-01-29 02:32:57 & 1355 & 4/11 & 17/196 & Faro De Finibus Terrae. ... \\
    2017-01-29 19:00:59 & 1262 & 6/11 & 36/196 & Buona domenica Igers... \\
    2017-01-23 04:15:08 & 1082 & 11/11 & 134/196 & Oggi... \\
    \bottomrule
  \end{tabular}
  \label{tab:exp1_4_top_5_posts}
\end{table}

図\ref{fig:exp1_4_jan}、表\ref{tab:exp1_4_top_1_posts}、表\ref{tab:exp1_4_top_5_posts}より、エッジの重要度が高いものは実際に多くの投稿に関連していることが確認できた。


\section{実験1.5: ノード特徴量の重要度とモデルへのスコア}
\subsection{目的}
重要度(importance)が高い要素ほど,その要素を操作したときのスコア変化(score impact)が大きい傾向にあるかを検証する.
これは「説明が予測に忠実である」ことの,学習不要な検証(入力介入による検証)である.

\subsection{手順}
\begin{enumerate}
  \item 実験1と同一の対象インフルエンサー $u$ を用い,説明対象月 $\tau$ を 1月,2月,3月に設定する.
  \item 各 $\tau$ に対して,特徴量のimportanceを得る.
  \item ベースライン置換を行い,score impact(符号付き影響度)を算出する(式\eqref{eq:score_impact_feature}).
  \item importanceとscore impactの関係を可視化し,整合性を確認する.
\end{enumerate} 

\subsection{結果}
図\ref{fig:exp1_5_jan}〜図\ref{fig:exp1_5_mar}に,1月,2月,3月それぞれにおけるimportanceとscore impactの関係を示す.
本図は,重要度が高いとされた要素が,実際にスコアに影響する要素として振る舞っているかを確認する目的で作成したものである.


\section{実験1.6: 重要な特徴がインフルエンサー全体でどのような効果を持つか}
\subsection{目的}
重要な特徴量がインフルエンサー全体でどのような効果を持つかを検証する。
具体的には、重要な特徴量を操作した場合に、その特徴量はインフルエンサー全体で有効なのかそれとも、特定のインフルエンサーにのみ有効なのかを調査する。

\subsection{手順}
\begin{enumerate}
  \item ランキング上位のインフルエンサーを対象に選択する(例:予測スコア上位 $N$ ユーザ).
  \item 各インフルエンサー $u$ に対して,説明対象月 $\tau$(1月,2月,3月)でのノード特徴量とエッジのimportanceを算出する.
  \item 行にユーザ,列に特徴量またはエッジグループ,値にimportanceを配置したヒートマップを作成する.
  \item ヒートマップを可視化し,以下を確認する:
  \begin{enumerate}
    \item 特定のインフルエンサークラスタにのみ効く特徴(局所的な重要度パターン)
    \item 全インフルエンサーに共通して効く特徴(汎用的な重要度パターン)
    \item 月次での重要度パターンの変動
  \end{enumerate}
\end{enumerate}


\subsection{結果}
ヒートマップの可視化結果を図\ref{fig:exp1_6_heatmap}に示す.
本図より,特定の特徴量がインフルエンサー全体で共通して重要であるのか,
それとも特定のインフルエンサーグループのみで重要であるのかを確認できる.

\begin{figure}[htbp]
  \centering
  \includegraphics[width=0.95\linewidth]{figures/exp/exp1-6-heatmap.png}
  \caption{実験1.6の結果:ランキング上位インフルエンサーの特徴量重要度ヒートマップ}
  \label{fig:exp1_6_heatmap}
\end{figure}


\section{実験1.7: Sanity Check(ランダム化テスト)}
\subsection{目的}
提案手法による重要度算出の仕組みが正常に機能しているかを検証する.
具体的には,モデル重みをランダム化した場合とラベル(目的変数)をシャッフルした場合における
importance分布とDeletion曲線が,元のモデルとどのように異なるかを比較することで,
説明手法が実際にモデルの予測機構を反映しているかを確認する.

\subsection{手順}
\begin{enumerate}
  \item 元のモデル(学習済み)を baseline として,対象インフルエンサー $u$ の月 $\tau$(1月,2月,3月)
  におけるimportanceを算出する.
  \item モデル重みをランダムに初期化したモデルを用いて,同一の対象・月に対してimportanceを算出する.
  \item ラベル(エンゲージメント率)をシャッフルして学習し直したモデルを用いて,importanceを算出する.
  \item 元のモデル,ランダムモデル,シャッフルモデルのimportance分布を重ねて可視化する.
  \item さらに,各モデルに対してDeletion分析を行い,上位重要度要素を削除したときの
  スコア変化曲線を比較する.元のモデルのみが明確に低下するはずである.
\end{enumerate}

\subsection{結果}
図\ref{fig:exp1_7_importance}に,元のモデル,ランダムモデル,シャッフルラベルモデルの
importance分布を並べて示す.図\ref{fig:exp1_7_deletion}に,各モデルに対するDeletion曲線を示す.
本結果より,提案手法が実際にモデルの予測に忠実な説明を生成しており,
ランダムまたは無関連なモデルでは重要度が出ない(または出ても無意味)ことが確認できる.

\begin{figure}[htbp]
  \centering
  \includegraphics[width=0.95\linewidth]{figures/exp/exp1-7-importance.png}
  \caption{実験1.7の結果:Sanity Check(importance分布の比較)}
  \label{fig:exp1_7_importance}
\end{figure}

\begin{figure}[htbp]
  \centering
  \includegraphics[width=0.95\linewidth]{figures/exp/exp1-7-deletion.png}
  \caption{実験1.7の結果:Sanity Check(Deletion曲線の比較)}
  \label{fig:exp1_7_deletion}
\end{figure}




% \section{実験2:重要度と実影響の整合性チェック(1月,2月,3月)}
% \subsection{目的}
% 重要度(importance)が高い要素ほど,その要素を操作したときのスコア変化(score impact)が大きい傾向にあるかを検証する.
% これは「説明が予測に忠実である」ことの,学習不要な検証(入力介入による検証)である.

% \subsection{手順}
% \begin{enumerate}
%   \item 実験1と同一の対象インフルエンサー $u$ を用い,説明対象月 $\tau$ を 1月,2月,3月に設定する.
%   \item 各 $\tau$ に対して,特徴量とエッジのimportanceを得る.
%   \item 特徴量はベースライン置換,エッジは除去を行い,score impact(符号付き影響度)を算出する(式\eqref{eq:score_impact_feature}).
%   \item importanceとscore impactの関係を可視化し,整合性を確認する.
% \end{enumerate}

% \subsection{結果}
% 図\ref{fig:exp2_jan}〜図\ref{fig:exp2_mar}に,1月,2月,3月それぞれにおけるimportanceとscore impactの関係を示す.
% 本図は,重要度が高いとされた要素が,実際にスコアに影響する要素として振る舞っているかを確認する目的で作成したものである.

% \begin{figure}[htbp]
%   \centering
%   \includegraphics[width=0.95\linewidth]{figures/exp2_jan.png}
%   \caption{実験2の結果(1月):importanceとscore impactの整合性の可視化}
%   \label{fig:exp2_jan}
% \end{figure}

% \begin{figure}[htbp]
%   \centering
%   \includegraphics[width=0.95\linewidth]{figures/exp2_feb.png}
%   \caption{実験2の結果(2月):importanceとscore impactの整合性の可視化}
%   \label{fig:exp2_feb}
% \end{figure}

% \begin{figure}[htbp]
%   \centering
%   \includegraphics[width=0.95\linewidth]{figures/exp2_mar.png}
%   \caption{実験2の結果(3月):importanceとscore impactの整合性の可視化}
%   \label{fig:exp2_mar}
% \end{figure}


\section{考察}
本章では,学習済みモデルを固定したまま実行可能な評価として,
(1) ランキング指標による予測性能の確認,


\section{まとめ}
本章では,学習済みモデルを固定したまま実行可能な評価として,
(1) ランキング指標による予測性能の確認,
(2) 実験1としてマスク値(importance)の分布確認(1月,2月,3月),
(3) 実験2として重要度と実影響(score impact)の整合性チェック(1月,2月,3月),
を実施し,説明が利用可能かどうかを検証するための結果を整理した.
次章では,これらの結果を踏まえて考察と限界,および今後の展望を述べる.
