\chapter{考察}
本研究では,InfluencerRank 型の時系列グラフ予測モデルの追試を行い,
学習済み予測器を固定したまま最適化ベースのマスク学習(GNNExplainer 系)を適用することで,
予測根拠(特徴量およびエッジ)を抽出し,その妥当性を検証した.
本章では,実験結果から得られた知見を整理し,本手法の限界と今後の展望を述べる.

\section{貢献度算出に関する考察}
\subsection{マスク値の分布と探索範囲削減としての有用性}
実験1では,月ごとに得られる特徴量マスク $m^X$ とエッジマスク $m^E$ を可視化し,
マスク値の分布が説明として利用可能かを確認した.
1月および3月の結果では,上位要素(特徴量・エッジ)が相対的に上位へ集中し,
top-$k$ や閾値 $\tau$ により候補要素を限定できる傾向が観測された.
このことは,特徴量次元やエッジ数が大きい設定において,
説明を重要候補の絞り込みとして用いる際に有用であることを示唆する.

一方で,1月および3月では,スコアへ影響を与える上位要素は検出できたものの,
それ以外の多数要素が0付近に張り付く傾向が強く観測された.
この現象は,最小構造を追求する損失の重み(疎性正則化)が強すぎる場合や,
探索空間が大きい状況での最適化の不安定性により,
説明が過度に疎になっている可能性を示す.
また2月の結果では,説明自体が安定して学習できていない可能性があり,
月によって最適化が成功しない条件が存在することが示唆された.

以上より,本研究の設定では,マスク値を精密な連続量として解釈するよりも,
条件を固定した上での相対比較(同一設定内での順位,top-$k$ の一致,閾値後の残存数)に寄せて用いるのが安全である.
特に月次比較では,重要度(mask値)の絶対値を直接比較するのではなく,
月内の順位や,後述のスコア差分(影響度)に基づく比較を主とする必要がある.

\subsection{重要度(mask値)と符号付き影響度(score impact)の整合性}
実験2では,重要度に基づく順位と,
実際に要素を置換・除去したときのスコア変化(score impact)に基づく順位の整合性を調べた.
説明が妥当に機能していれば,重要度が高い要素ほど $|\Delta|$ が大きい傾向が期待される.
この傾向が観測される場合,マスク最適化が予測を維持するために必要な要素を優先して残していることと整合する.

ただし,重要度と影響度が一致しないケースも生じうる.主な要因として以下が考えられる.
\begin{itemize}
    \item 相関した特徴量が多数存在する場合,どれか一つを残せば予測が保持されるため,重要度が分散または入れ替わる.
    \item マスク最適化は忠実性と疎性のトレードオフであり,影響度が最大の要素を必ずしも選ぶ最適化ではない.
    \item 局所解・初期値依存により,似た性能の別マスクへ収束する(多解性).
\end{itemize}

本研究の結果においても,月によっては重要度が0付近に集中しすぎるなど,
重要度だけでは説明の信頼性を担保しにくい状況が示唆された.
したがって,重要度のみで結論を出すのではなく,
スコア差分に基づく検証(重要度上位の要素を操作したときに予測がどの程度変化するか)を併用し,
さらに複数ユーザー・複数月で再現性を確認する必要がある.

\section{手法の限界}
\subsection{最適化規模とサンプル数の制約}
本研究では,説明のための最適化においてサンプル数が $30000$ 程度である一方,
最適化対象となるパラメータ(マスクの自由度)が $150000$ 程度に達する場合があった.
パラメータ数がサンプル数に比べて大きい状況では,
最適化が不安定になったり,多解性が強くなったりして,
得られたマスクの数値としての信頼性が低下する可能性がある.
その結果,探索範囲を絞る目的には有効でも,
マスク値を精密な定量量として扱うことには注意が必要である.
特に月によって最適化が失敗する可能性(2月の挙動)は,
この制約が現実的に影響した可能性がある.

\subsection{検証範囲の限定(単一ノード・少数時刻)}
本研究で詳細分析したのは,単一ノード(あるいは少数ノード)と少数の月に対する説明が中心であり,
一般性(他ノード・他期間・他条件でも同様に成立するか)は十分に検証できていない.
グラフ構造や投稿特性が異なるユーザー群では,
マスクの疎性や安定性,重要度と影響度の整合性が変化しうる.
よって,現時点の結論は限定された条件下での観察結果であることを明記する必要がある.

\subsection{相互作用(交互作用)を十分に扱えていない}
本研究では,一つの特徴量/エッジを置換・除去したときのスコア差分として影響度を定義した.
しかし,実際には複数要素の組合せによる相互作用(例:投稿頻度と特定関係性の同時変化)が
スコアに影響する可能性がある.
単一要素の差分だけでは,相互作用に起因する要因を十分に捉えられない場合がある.
また,相互作用が強い状況では,重要度が分散しやすく,
説明の再現性が低下する可能性がある.

\section{今後の展望}
\subsection{重要度順の追加・削除による反実仮想検証}
今後は,重要度順に要素を段階的に削除(または追加)し,
予測スコアがどのように変化するかを評価する実験を行う.
これにより,top-$k$ の要素が予測保持にどの程度寄与しているかを,
単発の置換よりも安定に検証できる.
具体的には $k=1,2,\ldots$ に対して,
上位$k$を削除した場合の予測低下と,
同数をランダムに削除した場合の予測低下を比較し,
説明がランダムよりも有効であることを確認する.
この評価は,0付近への過度な集中が観測された場合にも,
上位要素が本当に意味を持つかを直接検証できる.

\subsection{条件が近いユーザー群での再現性の確認}
説明の妥当性を受当性として主張するためには,
同様の条件を持つ複数ユーザーに対して,
類似の説明パターンが再現されることの確認が重要である.
例えば,投稿頻度・フォロワー規模・カテゴリなどが近いユーザー群を抽出し,
同一設定で説明を生成して,
top-$k$ の一致率や重要度と影響度の相関係数の分布を比較する.
これにより,説明の一般性と実務的な再利用可能性(パターン化)を評価できる.

\subsection{相互作用を考慮した貢献度の拡張}
本研究では単一要素の差分に基づく影響度を扱ったが,
今後は相互作用を捉える貢献度の設計が課題となる.
例えば,複数要素を同時に操作したときのスコア変化を用いる,
特徴量群・エッジ種別単位で集約した上で影響度を定義する,
または相互作用を評価するための反実仮想設計を導入することで,
単独では弱いが組合せで効く要因の抽出が可能になると考えられる.

\subsection{説明の安定化と評価指標の整備}
最適化に基づく説明は初期値依存・多解性が避けられないため,
複数初期値の平均・アンサンブルや,
得られた解の分散を併記するなど,
安定性を前提とした報告形式が必要である.
また,説明の評価指標として,
順位相関に加えて top-$k$ 一致率,閾値に対する残存要素数(疎性),
重要度順の削除曲線(ランダム比較),fidelity と疎性のトレードオフ曲線など,
説明の目的(絞り込みか定量解釈か)に対応した指標の整備が重要である.
