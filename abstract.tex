\documentclass[
  11pt, % 本文のサイズ
  twocolumn, % 二段組
  headings=small, % sectionなどを小さく
]{scrartcl}
\usepackage[
  left=1.5cm,
  right=1.5cm,
  top=1.5cm,
  bottom=1.5cm
]{geometry}
\usepackage[hiragino-pron]{luatexja-preset} % 日本語化
\usepackage[main=japanese,english]{babel} % キャプションなどを日本語化
\usepackage{setspace}\setstretch{0.9} % 字間を詰める
\usepackage{multirow} % 表を縦に結合
%%% テンプレートの設定の始まり
\pagestyle{empty} % 頁番号を削除
\setlength{\arrayrulewidth}{.8pt} % 罫線の太さ
\setlength{\tabcolsep}{3pt}
\setkomafont{title}{\normalfont\mcfamily} % タイトルを太字にしない
\title{ %%% 
  \vspace*{-30pt}
  \begin{minipage}[t]{1.0\linewidth}
    \begin{center}
      \LARGE
      修\ \ 士\ \ 論\ \ 文\ \ 概\ \ 要\ \ 書\\[3pt]
      \large
      Summary of Master's Thesis
    \end{center}
    \begin{flushright}
      \normalsize      
      Date of submission: \提出日
    \end{flushright}
  \end{minipage}
  \vspace*{-20pt} % タイトル下の空間
}
\author{
  \begin{minipage}[t]{1.0\linewidth}
    \begin{center} % 表の大きさは調整した方が良いかも
      \normalsize
      % \fontsize{10pt}{12pt}\selectfont % もう少し小さくする場合
      \begin{tabular}{|c|p{3cm}|c|p{3cm}|c|p{3cm}|}
        \hline
        \shortstack{\\専攻名 (専門分野)\\Department}&電気・情報生命&
          \shortstack{\\氏名\\Name}&\氏名& 
          \multirow[c]{2}{4em}{\shortstack{\\指導教員\\Advisor}}&
          \multirow[c]{2}{5em}{村田 昇}
        \\ 
        \cline{1-4}
        \shortstack{\\研究指導名\\Research guidance}&情報学習システム& 
          \shortstack{\\学籍番号\\Student ID\\number}&\学籍番号&&
        \\ 
        \hline
        \shortstack{\\研究題目\\Title}&\multicolumn{5}{l|}{\研究題目}
        \\ 
        \hline
      \end{tabular}
    \end{center}
  \end{minipage}
}
\date{\vspace*{-30pt}}
%%% テンプレートの設定の始まり

%%% 以下必要に応じて変更
%% 必要なpackageの読み込みの例
\usepackage{graphicx}
\graphicspath{{./figures/}}
\usepackage{amsmath,amssymb}
\usepackage[fleqn,tbtags]{mathtools} % amsmathの拡張
\mathtoolsset{showonlyrefs,showmanualtags} 
\usepackage{bxjalipsum} % ダミーの文書
%% 例の終わり
%% 
\def\氏名{渡邉 遼真}
\def\学籍番号{5324E110-4}
\def\研究題目{マルチモーダルなGNNにおける説明可能な影響力予測}
\def\提出日{02/29/2026}
\begin{document}
\maketitle
\thispagestyle{empty} % 表紙から頁番号を削除

%%% 以下本文の例
%%% (sectionだと大きいのでとりあえずsubsectionにしてみた)
%%% ===== 本文(2ページ想定)ここから:\maketitle の後に貼り替え =====

%%% ===== 本文(2ページ想定・分量増やした版)ここから =====

\subsection*{研究背景}
近年,スマートフォンおよびSNSの普及によりデジタルマーケティング市場は拡大しており,特にインフルエンサーマーケティングは企業の認知獲得や購買行動促進において重要な施策となっている.しかし,著名なトップインフルエンサーは起用コストが高騰し,競合との獲得競争も激しいため,早期に有望な人材を発掘し中長期で育成・契約する「青田買い」が実務上の鍵となる.

この意思決定を支えるため,SNS上の行動ログを用いて将来の影響力(例:エンゲージメント)を予測し,候補者をランキングする研究が進んでいる.一方で,多くの深層学習モデルは高精度でも予測プロセスがブラックボックスになりやすく,「なぜその人物が伸びると予測されたのか」という根拠が得られない.実務では,単に予測スコアが高いだけでは施策に結びつかず,投稿頻度・内容・関係性など,具体的に何が成長のドライバとなっているのかを説明可能な形で提示することが求められる.

そこで本研究は,時系列かつ異種混合(マルチモーダル)データに基づく影響力予測モデルに対して,予測と整合する説明を付与する枠組みを構築し,その妥当性と有用性を検証することを目的とする.

\begin{figure}[htbp]
    \centering
    \includegraphics[width=1\linewidth]{figures/thesis_position.png}
    \caption{本研究の位置付け(予測モデルと説明手法の統合,および月次比較可能な寄与分析)}\label{fig:thesis_position}
\end{figure}


\subsection*{問題設定}
本研究では,Instagramの投稿データ(画像・キャプション・ハッシュタグ等)とユーザー間の関係性(例:メンションやコメントの共起)を用い,月次の異種混合グラフ列を構築する.各時刻$t$におけるグラフを
\[
G_t=(V_t, E_t, \phi), \quad X_t \in \mathbb{R}^{|V_t|\times d}
\]
とする.ここで$V_t$はノード集合(ユーザー,タグ,オブジェクト等),$E_t$はエッジ集合(例:ユーザー–タグ,ユーザー–メンションなど),$\phi$はノード/エッジのタイプを表す.$X_t$はノード特徴行列であり,数値特徴(投稿数・投稿間隔・過去エンゲージメント統計)に加え,画像・テキストから抽出した特徴(例:画像輝度やテキスト長)など,マルチモーダル特徴を含む.

タスクは,過去$T$期間における入力$\{(G_{t-T+1},X_{t-T+1}),\dots,(G_t,X_t)\}$から,次期における各ユーザー$u$の影響力スコア$\hat{y}_u$を推定し,上位ユーザーをランキングすることである.本研究の目的は値の回帰精度よりも「有望ユーザーの上位抽出」であるため,学習はLearning to Rankとして定式化し,順位の整合性を直接最適化する.

なお,本データセットは完全なフォロー関係を含まないため,関係性はメンション等の代替指標により近似する.したがって,構築されるネットワークは「真の社会関係」を完全に反映するものではなく,観測可能な相互作用に基づく近似である.本研究ではこの制約を明示した上で,予測と説明の枠組みの検証を行う.

\subsection*{提案手法}
本研究は,(1)時系列・異種混合グラフに基づく影響力予測(InfluencerRank型)と,(2)最適化ベースのマスク学習による説明(MaskOpt)を統合する枠組みを提案する.目的は,単に「誰が伸びるか」を当てるだけでなく,「なぜ伸びると判断したのか」を特徴量および関係性(エッジ)レベルで提示し,月次比較可能な形に整えることである.

\paragraph{(1) 予測モデル:InfluencerRank型}
各月の異種混合グラフ$G_t$に対してGNN(例:GCN)を適用し,ユーザー$u$の表現$h_{u,t}$を得る.
\[
h_{u,t}=\mathrm{GNN}(G_t, X_t; \theta_{\mathrm{gnn}})
\]
次に,過去$T$期間の表現列$\{h_{u,t-T+1},\dots,h_{u,t}\}$をRNN(LSTM/GRU)で時系列統合し,隠れ状態$s_{u,t}$を更新する.
\[
s_{u,\tau}=\mathrm{RNN}(h_{u,\tau}, s_{u,\tau-1}; \theta_{\mathrm{rnn}})
\]
さらにTemporal Attentionにより各時刻の重要度$\alpha_{u,\tau}$を推定し,コンテキスト表現$c_u=\sum_{\tau}\alpha_{u,\tau}s_{u,\tau}$を得る.最後にMLPで将来スコア$\hat{y}_u$を出力し,このスコアでランキングを作る.この構成により,「どの月の状態が予測に効いたか」を注意重みとして解釈でき,後述の説明手法とも自然に接続できる.

\paragraph{(2) 説明手法:End-to-Endマスク最適化(MaskOpt)}
予測モデルを固定した上で,説明対象ユーザー$u$と説明対象時刻(または時刻集合)を定め,入力要素に連続値マスクを導入する.具体的には,
(i) ノード特徴量に対するマスク$m^{\mathrm{feat}}$,
(ii) 近傍エッジに対するマスク$m^{\mathrm{edge}}$
を学習し,マスク後の入力でも予測が大きく変化しない(忠実性)一方で,マスクが少数要素に集中する(簡潔性)ように最適化する.概念的には以下の目的関数を最小化する:
\[
\mathcal{L} = \underbrace{\mathcal{L}_{\mathrm{fidelity}}(\hat{y}_u^{\mathrm{masked}}, \hat{y}_u^{\mathrm{orig}})}_{\text{予測保持}}
+ \lambda_1\|m^{\mathrm{feat}}\|_1 + \lambda_2\|m^{\mathrm{edge}}\|_1
\]
最適化後のマスク値を重要度(importance)として用いる.重要度は「どの要素が選ばれたか」を表すが,符号(スコアを増やす/減らす方向)を含まないため,重要とされた要素を基準値へ置換したときの予測スコア差分を符号付き影響度(score impact)として算出する.これにより,「何が重要か(importance)」と「どう効くか(score impact)」を分離して提示できる.なお,score impactは局所的な反実仮想量であり,因果効果の断定は行わない.
\begin{figure}[htbp]
    \centering
    \includegraphics[width=1\linewidth]{figures/mask_optimization.png}
    \caption{MaskOpt の概念図(予測を維持しつつ特徴量・エッジを疎に残すマスクを最適化)}
    \label{fig:maskopt_concept}
\end{figure}



% \vspace{2pt}
% \begin{figure*}[t]
%   \centering
%   \fbox{\parbox{0.95\textwidth}{
%   \textbf{図:提案枠組み(プレースホルダ)}\\
%   月次グラフ$\rightarrow$GNN$\rightarrow$RNN/Attention$\rightarrow$ランキング予測.
%   予測モデル固定後,特徴量・エッジにマスクを導入して最適化し,importanceとscore impactを出力.
%   集約により「共通/個別」のパターンを可視化.
%   }}
%   \vspace{-6pt}
%   \caption{予測(InfluencerRank型)と説明(MaskOpt)の統合.}
% \end{figure*}

% \subsection*{応用例(実験)}
% \paragraph{データ分割と評価指標}
% 2017年1月〜11月を学習用,12月をテスト用として評価した.評価指標はランキング品質を表すNDCG@k(例:$k=10,50$)および閲覧行動モデルに基づくRBP(例:$p=0.95$)を用いる.比較として,単純ベースライン(過去統計に基づく順位付け等)および既存GNN系手法を用いた.

% \paragraph{実験1:予測精度の検証(定量評価)}
% まず,予測モデル自体が妥当なランキング性能を示すことを確認する.具体的には,テスト月における全ユーザーの予測順位と正解順位からNDCG/RBPを算出し,ベースラインや比較手法と相対比較する.この段階の目的は,後続の説明実験が成立する前提として「学習済みモデルが少なくとも妥当なランキングを出力している」ことを担保することである.

% \paragraph{実験2:予測根拠の可視化と妥当性検証(定性+忠実性)}
% 次に,予測上位ユーザーに対してMaskOptを適用し,重要特徴量および重要近傍(エッジ)を抽出する.さらに,上位要素を基準値に置換した際のscore impactを算出し,「重要とされた要素を操作すると予測が想定通り変化するか」を確認する.
% 加えて,説明がモデルの学習信号に依存していることを確認するため,重み・ラベルのシャッフル等によるsanity checkを行い,説明の一貫性が低下するかを検証する.

% \paragraph{実験3:一般性(共通/個別)の検討(集約)}
% 最後に,上位ユーザー群の説明結果を集約し,特徴量・近傍タイプごとのimportance/impactの分布を可視化する.これにより,(i)多くのユーザーで共通して重要となる要因,(ii)一部のユーザーにのみ重要となる局所要因,(iii)月次で変動する要因,を区別して整理する.

\subsection*{解析例}
ツールとしての信頼性を担保すべく、大きく3点の観点から解析を行った。
\paragraph{(1) 予測精度(定量)}
解析するにあたって,Instagramデータセット(約10万ユーザー,2017年1月〜12月)を用い,過去11ヶ月を学習用,12月をテスト用とした.評価指標はNDCG@50およびRBP(0.95)を採用し,単純ベースライン(過去エンゲージメント統計に基づく順位付け)と比較した.
表\ref{tab:perf}に主要指標を示す.InfluencerRank型モデルは単純ベースラインを上回り,将来有望ユーザーの上位抽出に有効であることを確認した.
また、図\ref{fig:model-prediction}に予測スコアのヒートマップ例を示す.
\begin{figure}
  \centering
  \includegraphics[width=0.48\textwidth]{figures/exp/exp1-model-prediction.png}
  \vspace{-6pt}
  \caption{集約例:importanceヒートマップ(プレースホルダ).}\label{fig:model-prediction}
  \vspace{-8pt}
\end{figure}

相関分析においても、予測値と真の値が、ずれている部分はあれど、一定の精度で予測できていることが図からわかる。


\begin{table}[t]
  \centering
  \vspace{-4pt}
  \caption{予測性能(例:12月テスト).}\label{tab:perf}
  \vspace{-6pt}
  \begin{tabular}{l|cc}
    \hline
    手法 & NDCG@50 & RBP(0.95) \\
    \hline
    提案(InfluencerRank型) & \textbf{0.xx} & \textbf{0.xx} \\
    ベースライン & 0.xx & 0.xx \\
    ランダム & 0.xx & 0.xx \\
    \hline
  \end{tabular}
  \vspace{-8pt}
\end{table}

この結果から、予測モデルが妥当なランキング性能を示すことを確認した。
フォロワー数、投稿頻度等の基本的な順位づけに加えて、他の影響度を予測する手法と比較しても高い性能を示すことから、インフルエンサー選定において有効なツールとなり得ることが示唆された。特に、過去のエンゲージメントの値を使用せずにコンテンツの特徴量や関係性に基づいて将来の影響力を予測できている点が重要である。
Pearson相関係数の値も0.48付近を示し、一定の予測精度が得られている。
よって、スコアの根拠を説明するための基盤として十分な性能を有していると判断した。

\paragraph{(2) 説明(定性+忠実性)}
分析の基となるモデルが妥当な予測を行うことを確認した上で,MaskOptを適用し,重要特徴量・重要近傍を抽出していく。出力結果の例として以下の図が得られる。

\begin{figure}[h]
\centering
\begin{minipage}[b]{0.49\columnwidth}
    \centering
    \includegraphics[width=0.9\columnwidth]{figures/exp/exp1-ego-vis.png}
    \caption{1月のエッジ重要度}\label{fig:ego-vis-jan}
\end{minipage}
\begin{minipage}[b]{0.49\columnwidth}
    \centering
    \includegraphics[width=0.9\columnwidth]{figures/exp/exp1-ego-vis2.png}
    \caption{3月のエッジ重要度}\label{fig:ego-vis2-mar}
\end{minipage}
\end{figure}
1月、3月のエッジ重要度可視化結果を図\ref{fig:ego-vis-jan},図\ref{fig:ego-vis2-mar}に示す。
この結果からそれぞれの月で重要な近傍エッジが異なることがわかる。
さらに、重要特徴量・近傍を基準値に置換した際のscore impactを算出し、説明の妥当性を定量的に検証した。
実際に過去の投稿で使用された際に高いエンゲージメントが得られているのか、全く重要性が感じられない要素が選ばれていないか等を確認し、説明の信頼性を担保した。

\subsection*{まとめ}
本研究は,(1)時系列・異種混合グラフに基づく影響力ランキング予測の追試実装と評価,
(2)特徴量・エッジを対象とするマスク最適化型説明(importance+score impact)の導入,
(3)月次比較可能な形での要因提示と集約による一般性分析,を行った.
実務的には,予測スコアに加え「根拠(要因候補)」を提示することで,施策立案や候補者選定の納得感・再現性向上に寄与することが期待される.
今後は,説明の安定性(初期値・ハイパーパラメータに対する再現性),基準値設定の依存性,
および特徴量とエッジの相互作用をより直接に扱う評価指標・可視化の整備が課題である.


%%% ===== 本文ここまで =====
\end{document}